\chapter{1877 G.W. Overprints}    

With the settlement of the dispute with the Orange Free State it was decided that Griqualand West should have its own postage stamps, and according to the early stamp journls they were to be of distinctive design. \textit{The Philatelist} stated that the order had been sent to England and that the design of the stamp was to be diamond-shaped! In the event however, the issue was more prosaic, merely an overprint on Cape stamps bearing the letters G.W. as shown in \ref{SG2}.

\ph[width = .30\textwidth]{../cape-of-good-hope/GW-overprint.jpg}{GRIQUALAND WEST 1877 1D UNUSED, RARE STAMP
1877 1d carmine-red (SG 2). Unused example. A rare stamp. Signed: Calves. (cv\pound700). \label{SG2}}

The first issue was of two values, 1d. and 4d., and the overprint consisted of the two initial letters: 'G.W.', in black on the 1d. and in red on the 4d. No record appears to exist as to the size of the setting; the largest block known being a pane of 60 of the 1d., which was in the A. H. West collection. Beyond saying that it was a setting of at least 60 and might have been twice as large so as to overprint two panes in one go. There were a few minor varieties in the setting consisting of a large full-stop after 'G' or after 'W". The 1d. exists with double overprint; only used examples being known, and these are very rare. The two stamps to receive the overprint were the 1d. carmine-red (without outer frame) and the 4d. blue (also without outer frame). 

\ph[width = .90\textwidth]{../griqualand-west/13027_733_1.jpg}{733 1877 (Mar.) overprinted "G.W." 
1d. carmine-red and 4d. blue unused or with part
original gum, and used; the unused 4d. with a thin, the used a little soiled, otherwise
sound. S.G. 2, 3, \pound1,280 \pound150-200.}

The overprinting was almost certainly the work of Messrs Saul Solomon \& Co., the Cape Government Printers, and the two values were issued in March 1877.  

            