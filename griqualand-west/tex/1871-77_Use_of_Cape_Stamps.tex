\chapter{The use of Cape Stamps without Overprint, 1871-77}    


Shortly before Griqualand West was proclaimed British territory in 1871, the Cape
Government instituted a postal service, opening post offices at which ordinary Cape stamps without overprint were sold. A contemporary account of the conditions at the time was published in \textit{Stamp-Collector's Magazine} for 1874 (Volume XII, p. 63), as a communication from E.L. Pemberton:

\begin{blockquote}

\textit{Posts at the Diamond Fields.}- There are government post-offices at the principal places on the Fields (Klip Drift, New Rush, and Du Toit's Pan), which are conducted exactly the same as those in the Cape Colony, the same rates of postage are charged, and money-orders are issued on all parts of the colony and the United Kingdom at the same charge. No special stamps are yet issued for Griqualand, those of the Cape being used at present; but as the governments of the who are now different, probably a special issue will be made for the former, though no steps have as yet been taken in that direction. Bi-weekly mails run between the colonies and the fields, conveyed by contractors in the usual manner in South Africa--a cart and four. A weekly mail is also despatched from Klip Drift to places further up the interior, and there is a daily one between the above-mentioned places.

In addition to the government post-offices, there are, at both New Rush and Du Toit's Pan, what are called 'Natal and Free State Post-Offices.' These are the property of private persons, and are simply agencies for forwarding and receiving letters to and from a post-office on the borders of the Orange Free State. Most Natal letters pass through this office, as it is the quickets way of sending them; and the proprietors of the agency charge a monthly subscription of two shillings and sixpence, which entitles the subscriber to send and receive as many letters as he pleases: non-subscribers have to pay sixpence for each letter sent or received through the agency. The stamp required on letters to any part of Natal or the Orange Free State by this means is 6d., O.F.S. The reason for the stablishment of this agency was, that letters sent through the government office would have to pay 4d. Colonial and 6d. Orange State, which is the rate paid in colonial towns, as there is no postal convention between the two countries.

\end{blockquote}    


In September 1870 tenders were invited to convey the mail from Hopetown to Klip Drift, which has been opened following a petition of its residents<sup>2</sup>. As Goldblatt notes the petition was not surprising as the Hopetown residents had to cross the Vaal River by ferry and pay a fee of 2s in order to collect their mail from Pniel. 

Pniel (pronounced Peeneel) is situated on the east bank of the Vaal River, about 100 miles from its mouth. The Town attracted early diamond diggers with the diggings being about three miles from the mission station. The Vaal at the diggings is about 200 yards wide, and numerous yawl ferries are constantly plying to and from Klip Drift, or Parketon as it was called. A large wagon ferry has also been started, to be used at high water, when the river cannot be forded.

\begin{blockquote}
There are about 3,000 inhabitants at Pniel at present; and although many leave for the new rushes, teh place is gradually growing\ldots

A post-office has been established, and a newspaper, called the \textit{Diamond News}, is successfully under way, and has six pages of new and advertisements; it is issued weekly (on Saturday). You can safely ship diamonds by post from here if you register them, and you can rely upon their being delivered to any part of the colony that you may wish to send them.
\end{blockquote}

\subsection{References}

2 \goldblatt{178}

South African Diamond Fields, J. L. Babe, Correspondent of the "New York World." David Esley & Co., 1872.

          