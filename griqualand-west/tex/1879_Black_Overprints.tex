\chapter{1879 Black Overprints}    
The last overprinting before Griqualand West was absorbed by Cape Colony took place
in 1879, a date given by Holmes with some reserve, as no stamp of this issue has been seen with an earlier date than that year. As before, the setting was one of 120, and this time the type consisted of ordinary 8-point roman capital G's. The overprinting was in black.
\ph[width = .98\textwidth]{../griqualand-west/griqualand-block.jpg}{
5 Sh, block of nine, three stamps as well in the landscape format, 
postmark "227" KIMBERLEY. Rare large unit (street), 
certificate with photograph Knopke. (Catalog value: 1 530)
EURO 600
}

Six values were overprinted:

\begin{tabular}{llllll}
 &1/2. grey black & & & &\\
 &1d. carmine red.& & & &\\
 &4d. blue (without frame line).& & & &\\
 &6d. mauve.& & & &\\
 &1s. green.& & & &\\
 &5s. orange-yellow.& & & &\\
\end{tabular}

As before all are watermarked Crown over CC and perforated 14. Holmes reports that a complete setting of this issue has not been see, the largest blocks available being an entire left pane of the \half d. and a block of seventeen 1d. from the right pane. Some of the G's in the setting are of a different font, being similar to the antique G's of the previous setting. Two are to be found in the left pane, Nos. 20 and 58. Blocks of all values of this issue, it should be emphasiszed, are exceedingly rare.

All six values are known with double overprint, and the 1d, and 5/- with tremble overprint. These multiple overprints are different from the usual run of such things in that they were applied purposely to make the overprint more distinctive.

\ph[width = .50\textwidth]{../griqualand-west/AC624.jpg}{AC624	1879 5/- yellow-orange with interpanneau margin at left, overprinted 'G' (Type 17). Showing the variety OVERPRINT DOUBLE. Few light hinge remnants, otherwise very fine mint. Stanley Gibbons handstamp on reverse. Rare. SG 29a, McGregor stock.}

Inverted overprints occur on the 1d. and 6d. The 4d. is also said to exist inverted, but no specimen has been seen.




          