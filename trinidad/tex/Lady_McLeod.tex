\section{Lady McLeod}

The fabled Lady McLeod stamp was the first adhesive issued 
in a British Colony in April of 1847. It was issued by David 
Bryce for prepayment of mail carried between Port of Spain 
and San Fernando aboard his paddle steamer "The Lady McLeod". 
There is no value expressed but the stamp paid a fee of 5c. 
The stamps were generally cancelled by means of a pen-cross.
 

\ph[width=.20\textwidth]{../trinidad/2173.jpg}{
2173
SG1 1847 (5c) blue imperf Lady McLeod, deep shade, unused. 
Three small but adequate margins. Pressed crease, o/w a sound and 
very presentable example. Marriott number M16. Fine. 
2009 RPS certificate. (Scott \$50,000, SG \pound28,000). 
\$  7,500
}

\ph[width=.20\textwidth]{../trinidad/2174.jpg}{
2174
SG1 1847 (5c) blue Lady McLeod unused. Four margins. 
Extensively repaired at LL and thinned. Very attractive appearance. 
Red hs on reverse. Fine. Ex Burrus, Turner. 
Marriott number M4. (Scott \$50,000, SG \pound28,000). 
\$  2,500
}

\ph[width=.95\textwidth]{../trinidad/2175.jpg}{
2175
SG1 1847 (5) Lady McLeod on complete folded letter 
to Port of Spain headed San Fernando 30th August, 1847. 
The stamp is affixed at the UL corner and cancelled with an "X". 
Ample to large margins with portion of adjacent stamp visible at right. 
Very fresh and a beautiful example, unquestionably one of the 
finest extant. File fold 
does not affect stamp. Ex Snowden, Hurlock. 
Marrott number C17. 
}







                                            