\section{Bophuthatswana, Stamps and Postal History}

Bophuthatswana (meaning "gathering of the Tswana people"),[1] officially the 
Republic of Bophuthatswana (Tswana: Repaboleki ya Bophuthatswana; Afrikaans: 
Republiek van Bophuthatswana), was a bantustan ("homeland"; an area set aside for members of a specific ethnicity) and nominal parliamentary democracy in the 
northwestern 
region of South Africa. Its seat of government was Mmabatho.

Historically, Bophuthatswana's significance is twofold: it was the first area to be declared an independent state whose territory constituted a scattered patchwork of individual enclaves, and during its last days of existence, events taking place 
within its borders led to the weakening and split of right-winged Afrikaner 
resistance towards democratizing South Africa.

In 1994, it was reintegrated into South Africa, and its territory was distributed 
among the new provinces of the Orange Free State (now Free State), Northern 
Cape, and North West Province.

\ph[width=.95\textwidth]{../bophuthatswana/first-set.jpg}{Bophutatswana - Yvert 1/17 \$5.00
}


                                      