\section{Octagonal Re-issued}

The Octagonal handstamps were pressed again into service after being withdrawn. Rectangular postage stamps defaced by them can be found on covers at relatively late dates. No. 51 of Port Alfred appears in February 1877, and No. 30, Swellendam, as late as 1890. Strikes of the Octagonal Numeral Stamp used both as a dispatching and a receipt mark on the same cover are known, but are rare.

\ph[width = .80\textwidth]{../cape-of-good-hope/re-issued-octagonal.jpg}{
1879 SWELLENDAM to USA: Neat small envelope franked 7 1/2d made up with 7x 1d's and a 1/2 d all tied by strikes of REISSUED OCTAGONAL HANDSTAMP 30. Cape Town cds JA 9 79, London transit, San Fransisco arrival cds. Some overlapping of stamps, F-VF, very attractive example of rare late use octagonal handstamp, nice destination. ..nssng/056	
Price: \$ 345
}

\ph[width = .80\textwidth]{../cape-of-good-hope/octagonal-reissued.jpg}{1893
 cover to London bearing 1d. red tied by two strikes of octagonal framed "51" numeral with "PORT/ALFRED" c.d.s. alongside. A scarce late use of this numeral cancel.
}

Other rectangulars have been recorded cancelled by this stamp, but these appear to have been stamped during their normal use. Nos. 10, 13, 17, 30, 40, 46, 51, 53 and 59, but with the exception of "51", where its use as a defacing handstamp was sanctioned, the late use of others would appear to be totally unofficial.

                