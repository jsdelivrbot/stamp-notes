\section{Briefstock Letters}

 
Early local correspondence in the Cape of Good Hope was transmitted 
using letter runners. 
These were usually Hottentot servants. 
They wore only a loincloth and could therefore not carry letters 
on their person. 

\ph[width = .60\textwidth]{../cape-of-good-hope/geo1113shd18p5.png}{
1785 Small format folded entire addressed in High Dutch to Monsr 
Monsieur J W Martin op Zijn Plaats aan der Bottelarij. Dated 2 January, 
the writer being P De Villiers of Tygerberg. Shows central vertical folds 
and soiling pattern associated with being placed in the cleft stick of 
the carrier. An outstanding example of this rare mail. Provenance: Goldblat
}

A cleft stick, the so-called \textit{briefstock}  was used for this purpose. 
The briefstock varied from one to two metres in length and the top 
was split down the centre to a depth of about 200 mm. The bottom end 
of the split was bound 
with a {{dfn:riempie}} (rawhide thong) preventing further splitting 
when letters were inserted. To hold the letters the top part was 
again tightly bound with another riempie.

These early Postal History items of the Cape of Good Hope can be 
identified by the indentation
or crease mark in the middle, made by the edge of the stick.
 
\wrapright[width = .30\textwidth]{cape-of-good-hope/briefstock-runner.png}{ }
These Hottentot servants had an amazing stamina. They have been 
reported that they could easily cover 80 to 90 km per day at a 
steady jog-trot, holding the stick in front of them. The stick served 
a dual purpose, for not only it prevented the runners from handling 
the letters, but it also protected the letters from getting wet 
when the runners were crossing streams. This method of carrying 
letters was not confined to the Cape of Good Hope but was also 
reported elsewhere in southern Africa and as far as Japan.

This method of carrying briefstock letters was reported in Matabeleland 
as late as 1887. It has been noted that Sir Sidney Shippard, the 
Administrator of British Bechuanaland, while on his journey to 
the Matabele chief Lobengula, is known to have dispatched a 
letter by runner to the Assistant Commissioner, Mr. Moffat at Gubuluwayo.

A limited number of these briefstock covers exist. You should be very careful 
if you buying any of these items as a number of forgeries exist. 
If you purchase one try 
and obtain a certificate and try and ascertain its provenance if possible.

                                  