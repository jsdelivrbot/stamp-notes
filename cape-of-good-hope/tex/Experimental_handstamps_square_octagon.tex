\section{Square Octagon.}



\ph[width = .90\textwidth]{../cape-of-good-hope/square-circle/square-octagon/square-octagon-cover.jpg}{
13 DEC 1888 Cover from Cape Town  to Worcester, bearing two seated Hope 1d red 
and handstamped Square Octagon handstamp  (Goldblatt  SO 1) 
datestamp dated 13 Dec  88, Control letter N  
backstamped Worcester cds DEC14 88.
}


The Square octagon datestamp  (Goldblatt SO  1 ) was another 
experimental postmark aimed in finding a canceller that could 
be used for the simultaneous defacing and dating of letters. 
It followed the issue of the square circle postmarks. The Square 
Octagon canceller was issued to the Cape Town General Post 
Office exclusively in 1887. The datestamp resembles those 
used in the Orange Free State and the South African Republic
(Transvaal) at the time.

It appears that the earliest date of use was 1st May 1887 
on a cover from Cape Town to Worcester. An example with dark 
blue ink has also been recorded. 

This postmark  had a very short lifetime and was later 
replaced by the Hooded Circular Datestamp of 1888. It must be 
noted that about the same time the British Post Office was 
carrying out similar experiments.

                                         