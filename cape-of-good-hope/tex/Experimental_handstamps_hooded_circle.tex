\section{The Hooded Circular Datestamp of 1888}

\ph[width = .90\textwidth]{../cape-of-good-hope/SQUARE-CIRCLE/HCDS/HoodedPortElizabeth.jpg}{ }


This datestamp was another experimental postmark issued in an attempt to 
develop a combined defacer and backstamping canceller. It was issued 
to the larger post offices in the Cape of Good Hope in 1888. This 
stamp did not gain general favour and was never issued to 
smaller post offices.

The Hooded datestamp derives its design from similar datestamps 
of Great Britain. It consists of a circle with a diameter of 19 mm, 
surmounted by a 'hooded' portion which contains the name of the town. 
The circle is divided into three sections by two horizontal bars. 
In the middle of the bars is the date and the top and bottom have 
either a star or a time control letter or none, which results in 
the different variations of this stamp. Six variations are 
normally recognised and these are designated HCDS 1 to HCDS 6. 

Variations of the postmarks are normally recognised and these are 
designated \textsc{HCDS 1} to \textsc{HCDS 6}. 

\begin{comment}
<table align="center" cellpadding="2" cellspacing="0" width="579" border="0">
   <tbody>
  <tr>
   <td width="151"><img src="http://localhost/capepostalhistory/SQUARE-CIRCLE/HCDS/HCDS.jpg" alt="HCDS1" height="130" width="151"></td>
   
   <td width="158"><img src="http://localhost/capepostalhistory/SQUARE-CIRCLE/HCDS/HCDS.jpg" alt="HCDS 2" height="130" width="151"></td>
   
   
   <td width="127"><img src="http://localhost/capepostalhistory/SQUARE-CIRCLE/HCDS/HCDS.jpg" alt="HCDS 2" height="130" width="151"></td>
   <td width="133"><img src="http://localhost/capepostalhistory/SQUARE-CIRCLE/HCDS/HCDS.jpg" alt="HCDS 2" height="130" width="151"></td>
  </tr>

  <tr>
                <td><div style="font-weight: bold;" align="center">HCDS 1 </div></td>
                <td><div style="font-weight: bold;" align="center">HCDS 2 </div></td>
                <td><div style="font-weight: bold;" align="center">HCDS 3 </div></td>
                <td><div style="font-weight: bold;" align="center">HCDS 5 </div></td>
  </tr>
  <tr>

                <td><div style="color: rgb(0, 0, 255);" align="center"><span class="ImageDescription" style="font-style: italic;">Star on top, control letter belo</span><span class="ImageDescription">w </span></div></td>
                <td>Vice versa to HCDS 1 </td>
                <td><div align="center">Control letter in the Upper portion and blank below. HCDS 4 is vice versa to this. </div></td>
                <td><div align="center">Blank on top star at bottom. HCDS 6 is vice versa to HCDS 5 </div></td>
  </tr>
  </tbody>
</table>
\end{comment}
                                                          