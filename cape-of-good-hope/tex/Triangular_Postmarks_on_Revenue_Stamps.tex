\chapter{Revenue Stamps Used Postally}

In the \textit{The Philatelic Record} p.68 1884, it states:

Cape Of Good Hope.-- We have seen the following fiscals, head of Queen in inscribed circle, used postally : 2s. 6d. green, 4s. lilac, 8s. ultramine.




\ph[width = .80\textwidth]{../cape-of-good-hope/jurgens-01.jpg}{ }

	 
Occassionally Revenue stamps of the Cape of Good Hope are 
found on covers or pieces 'postally used'. Both Allis{{footnote:1}} 
as well as Jurgens{{footnote:2}} alluded to this fact. Robson Lowe 
in his Encyclopaedia of British Empire Postage Stamps, Volume II, 
The Empire in Africa, page 34, First Edition March 1st, 1949, writes:

\begin{blockquote}
... Between 1874 and 1875 Revenue stamps were used postally 
and are to be found on covers. They are the embossed 1 d. 
brown and 6d. blue and the de la Rue 1d. watermark CC. 
\textit{All are rare}. (The emphasis is my own).
\end{blockquote}

 
He also has an illustration shown. It is the considered 
opinion of most Cape of Good Hope philatelists that all 
these covers were as a matter of fact forged by Jurgens, 
who wrote about them extensively in his book, quoted above. 
It also appears that Robson Lowe's primary source for this 
information was Jurgen as he writes:

\begin{quotation}
... Special acknowledgement is made to Mr. A.A. Jurgens, 
whose standard work on the "Handstruck Letter stamps of 
the Cape of Good Hope has been the source of most of the 
information given above.
\end{quotation}  
 
The illustation shown here, is I believe in Jurgen's own 
hand and I presume that he 
has sent it to Robson Lowe himself. 

The way the forged handstamps were exposed was described in an unrelated
article by Robson Lowe<sup>3</sup>.

\begin{quotation}
\ldots \textit{Dr J. Harvey Pirie} was a doctor on an Antartic
Expedition in 1900. He became the President of the South African
Federation and was a notable philatelist. Pre-war, I had an adventure
with him at a Johannesburg Exhibition. The postal history exhibits had
aroused my interest and I studied them close noticing a handstamp
in shocking pink from the Cape of Good Hope, and another in the same
colour from the Orange Free State. In the mid-19th century, this colour was
not known, and I drew Harvey Pirie's notice to this peculiarity. In consequence
the forged handstamps made by A.A. Jurgens were exposed by my old
friend, Douglas Roth\ldots
\end{quotation}

 
Adrian Albert Jurgens (1886-1953) was a South African 
philatelist and signatory to the Roll of Distinguished 
Philatelists (RDP) in 1952.

In 1944 Jurgens won the Crawford Medal from the Royal 
Philatelic Society London for his work The Handstruck Letter 
Stamps of the Cape of Good Hope from 1792 to 1853 and the 
Postmarks from 1853 to 1910.


It appears that later on Robson Lowe discovered the 
fact \url{http://www.robsonlowe.co.uk/} and he famously declined 
to sign the RDP due to the organisers failure to 
delete Jurgens name. Lowe regarded Jurgens as a forger. 

Jurgen's main area of interest was the philately of 
Southern Africa, in particular the Bechuanalands and 
Cape of Good Hope, and the A.A. Jurgens - Cape of 
Good Hope, Barbara Jurgens Memorial Collection of 
twenty volumes is in the Iziko Museums of Cape Town.[3] 
Barbara was Adrian's daughter.

This piece of postal history has many angles pointing 
to an outright fake. If Adrian did it to prove a point 
or to profit is unknown. Since his collection was eventually
donated to a museum I tend to lean on the 'vanity' theory. 
Any information would be greatly appreciated.

He is still listed on the Roll of Distinguished 
philatelists on the \href{Philatelic Federation of 
South Africa (1948).}{http://www.philatelysa.co.za/RDPSAS.html}

The item below was auctioned on ebay on 14 July 2012, I am not sure 
if it was a Jurgen's fake but it has all the signs of it. 

\url{http://www.ebay.com/itm/ws/eBayISAPI.dll?ViewItem&item=380452270869&ssPageName=ADME:B:WNA:US:1123\#ht_500wt_823)}

It fetched a price of \pound;800 ++ I stopped looking when it hit 730 pounds.

\ph[width = .95\textwidth]{../cape-of-good-hope/fakes/revenues.jpg}{ }

\ph[width = .95\textwidth]{../cape-of-good-hope/fakes/jurgens-01.jpg}{Jurgens faked cover with triangular obliterator and faked DTO Swellendam. Note both covers have the same date, MAR 27, 1867. The cover is from an ebay auction, which unfortunately I did not keep the details. The seller was from the Cape.}



1. Robson Lowe, \textit{Philatelic Personalities, A Glance At the Philatelic Past},
London Philatelist, December 1993 102:353. (LP1211.pdf)
 
 

                        