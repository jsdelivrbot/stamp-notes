\section{The King Edward VII Stamps of 1902
} 


The relationship of the Cape Colony to the Mother Country was for the first time indicated upon what proved to be the last issue of Cape of Good Hope postage stamps, namely, those which appeared at intervals from 1902 to 1904, bearing as the centre-piece the profile portrait of His Majesty King Edward VII The design is said to have been adapted from the same die as that employed in the preparation of the plates from which the King Edward stamps of Great Britain were prepared.

\ph[width = .70\textwidth]{../cape-of-good-hope/frank/385.jpg}{ 
Sale 12031 Lot 385

1902-04 King Edward VII Issue 
Collection predominating in a fine range of mint multiples, comprising 1/2d. (4, one with one stamp partially imperforate due to paper fold), 1d. (4), 2d. (2), 2 1/2d. (4), 3d. (2), 4d. (3), 1/- (4) and 5/- blocks of four, 1d. (2) and 6d. blocks of six, 2d. and 4d. blocks of twelve, 3d. and 5/- block of eight, the used with 5/- blocks of four (2), and covers (7) with a Parcel Post receipt bearing 1d., one to Constantinople at 2 1/2d. and one registered to Cavalla, Turkey at 6 1/2d. Generally fine and an impressive lot. S.G. 70-78, \pound;3,440+. Photo 
Estimate \pound; 1,000-1,200
}



"Hope," having risen from her seat after a period of contemplation lasting for upwards of four decades, having stood, at intervals, upon the One Penny, Halfpenny and Three Pence values, for seven years, and having finally appeared above the clouds on the One Penny stamp of 1900, now entirely disappears.

\subsection{Imperforate Plate Proofs
} 
Imperforate plate proofs exist for all values. The images on the left and bottom show some of the colour variations that one can find for these plate proofs.

\ph[width = .50\textwidth]{../cape-of-good-hope/ADHESIVES/edward-proofs.jpg}{
Sale 7034 Lot 70

Cape of Good Hope
1902-04 King Edward VII Issue
Imperforate Plate Proofs
1/2d. to 5/- set of nine in horizontal pairs on gummed watermarked paper; a couple with creases, otherwise fine and a most attractive group. Photo
Estimate \pound; 1,200-1,500 sold 2900. Spinks sale Frankschoek, June 2012
}

The new stamps of the Cape of Good Hope, of nine denominations, from Half-penny to Five Shillings, were printed by Messrs. De La Rue \& Co. who also prepared the dies and plates. All values were printed upon the usual paper with the cabled anchor watermark normally used for Cape of Good Hope stamps.


\ph[width = .98\textwidth]{../cape-of-good-hope/frank/384.jpg}{
Sale 12031 Lot 384

1902-04 King Edward VII Issue 
1/2d., 1d., 2d., 2 1/2d., 3d., 6/-, 1/- and 5/- imperforate 
plate proof horizontal pairs in issued colours on gummed 
watermarked paper; 6d. and 1/- with crease though a scarce group. 
Estimate \pound; 1,200-1,500
}
						
\ph[width = .98\textwidth]{../cape-of-good-hope/ADHESIVES/King-Edward.jpg}{
The full set.
}

The central portrait of King Edward VII is the same throughout. 
Unlike other colonial stamps each value has its own distinctive 
frame of an ornamental pattern. In this frame the words 'Cape of 
Good Hope, postage," and the value are incorporated.

Several printings were made resulting in a variety of colour 
shades in all values.
For the One Penny stamps four different plates were needed, 
whilst three plates sufficed to provide the necessary supplies 
of the Half-penny value. In all other cases one plate only was employed.
 

In certain of the sheets of the One Penny stamps the horizontal 
lines of shading in the medallion containing the King's portrait to 
the right of the head, are almost entirely absent This is probably 
owing to the plate being worn, or to uneven inking. Another whilst 
minor variety exists in the same value where a small spot of colour 
resembling a full stop occurs between the letters " ?" and "n" of " one."
Cape of Good Hope: Postal History: King Edward VII Plate Proof
 

Other values appeared from time to time up to October 1904, when the series was completed by the issue of the Two Pence value.


Considerable stocks of the stamps of old designs were still on hand in Capetown, and during certain periods the King's Head types were withdrawn to admit of the old stocks being used up, whilst at other times, the old and the new stamps were in concurrent use.
 
 
Official Stamps

In 1perfin906 the Cape Government Printing and Stationery Department adopted a system of perforating in connection with the stamps used by their Department upon their Foreign Mail matter. The stamps so used were first passed through a machine and a device, consisting of eleven round holes, in the form of two triangles, having their common apexes meeting in the hole at the centre of the stamp, was punched out, thus:

Definite information as to all the denominations of the stamps so treated is not available, but the following have been seen : One Penny, Two Pence, Three Pence, Four Pence, Six Pence, One Shilling and Five Shillings.


official King Edward VII
 

Other values of the King Edward stamps may be found with various forms of overprint for revenue purposes, and were employed in the collection of cigarette and customs duties. Although these frequently bear postal obliterations, they were never used for postal purposes, and have no real philatelic interest.
Privately Perforated Stamps
perfins
 

Stamps are found privately perforated normally with the Company initials to eliminate pilferage of stamps from the Company concerned.
Stamps are found for John Garlic, Cape Times Ltd., Union-Castle Steamship Co. Lt., C E Gardner \& Co. Ltd., R Muller

 

 
REFERENCE LIST

Rectangular design with portrait of King Edward VII. Watermarked Cabled Anchor. Perf. 14.  Medium white wove paper.  White gum.
half penny. Issued December 1902. In shades of green.


Plates i, 2 and 3.4


one penny. Issued December 1902. In shades of rose-red and carmine. Plates 1, 2, 3 and 4.
Two pence. Issued October 1904. In shades of brown.
twopence halfpenny. Issued March 1904. In shades of ultramarine-blue.
three pence. Issued April 1903. In shades of magenta.
four pence. Issued February 1903. In shades of sage-green.
Six pence. Issued March 1903. In shades of lilac and mauve.
one shilling. Issued December 1902. In shades ofyellow-ochrc.
five shillings. Issued February 1903. In shades of orange-yellow.
note. All the above stamps exist imperf. and are known in imperf. pairs. In this condition they are from plate-proof or trial sheets.

New Issue of Postage Stamps. - Upon the accession of His Majesty to the Throne it was decided by the Government to introduce new designs of postage stamps, post cards, etc., being a representation of the head of the King. Instructions were
accordingly issued to Messrs. Thomas De la Rue and Sons, London, for the preparation of suitable designs, and these were in due course submitted and approved. They consist of oval centers, showing the head of His Majesty in profile (after an original drawing by Mr. Emil Fuchs,the sculptor), set in suitable frames. His Majesty having commanded that all stamps bearing his effigy should have an
Imperial crown in the border, this has been adopted, and the crown appears in the new issue immediately above the head, adding considerably to the appearance of the stamps. In the upper corners the values of the stamps are indicated in bold
figures; and in words at the base.


It was Messrs. De La Rue's proposal to have the same frame for each denomination of adhesive stamp; but, in view of the resemblance between the colours of certain denominations, more particularly in artificial light, and that something more than a more colour difference is therefore needed to enable the several values to be readily distinguished under all circumstances, I recommended a different design of frame for each value, and this suggestion has been adopted. As the colours of the stamps are fixed by the Universal Postal Union Convention, it was not possible to make any alteration in that respect, nor could any such alteration have been entirely effective, but the adoption of the different frame for each variety will remove any likelihood of mistake in those cases where the colours approximate, and will distinctly facilitate the checking
of postage.
King-Edward

It was hoped that the new stamps would be ready for issue on the 1st January, 1902, but owing to the quantity of similar work for other Colonies which the contractors have in hand, and the great length of time which it was found would
be occupied in engraving in turn the various dies and formes, this has proved impossible. It is anticipated, however, that the new dies will be brought into use in connection with the execution of the next indents for supplies; in which case some of the denominations will be put into circulation towards the close of the year 1902 (1901, p. 29).


\ph[width = .70\textwidth]{../cape-of-good-hope/protea-sugar-bush.jpg}{The 2\halfd frame incoporates, a protea flower in the frame design.}

De La Rue were unable to find a picture of a Protea to incorporate in their design. As a result the Cape Agent General sent a copy of the \textit{Curtis Botanical Magazine} to Sir Evelyn Andros De La Rue. The note from the Agent General refers to page 346 and asks to be returned as it was borrowed!



http://djvued.libs.uga.edu/QK1xC981/1f/curtis_botanical_mag_vol_09_10.pdf (Protea for design)

\subsubsection{References}
1. Trotter Brian, Cape of Good Hope Edwardian Postage Stamps, \textit{London Philatelist}, July/August 2001, \textbf{110}:198.

1287.pdf


                                                           