
\section{An Introduction
} 
For collectors  the \href{../general/postal_history}{postal history} and 
stamps of Cape of Good Hope are entrenched in mystery and romance. 
The mixture of explorers, travellers, wars, the birth of a nation and the birth of 
a new country make for a fascinating past. 

My name is Dr Yannis Lazarides and South Africa has been my home for the 
last twenty six years. Since I was a child I was fascinated by the Cape of Good Hope, 
its stamps---out of reach then and some of them out of reach now---and its postal history. 

I have put this collection of pages on the internet to promote the study of 
the _Postal History_  of the  _Cape of Good Hope_. 
The postal history images presented here are mostly based on my own collection. Here you will find a mixture of information on virtually every aspect of the Postal History of the Cape of Good Hope and to a large extend its people and its history. 

The history of the Cape of Good Hope, 
starts with the discovery of the Cape by Portugese sailors, the settlements by the <a href="the%20dutch%20in%20the%20Cape%20of%20Good%20Hope.html">Dutch</a> beginning with <a href="Janvanriebeeck.html">Jan van Riebeeck </a>and the VOC Company. A string of commanders and <a href="voc-capegovernors.html">VOC Governors</a> ruled the Cape of Good Hope for almost 150 years. early correspondence is covered with some rare<a href="Briefstock%20Letters.html"> briefstock</a> and <a href="voc-handstamp.html">VOC letters</a>. The subsequent colonization by the British is covered by the <a href="second-british-occupation.html">First British Occupation</a> and the <a href="second-british-occupation.html">Second British Occupation</a>. This period saw the development of postal services and the introduction of various postmarks. Of particular interest are the <a href="crown-in-circle%20handstamps.html">crown-in-circle</a>, the rare <a href="Port-Elizabeth-topay-paid.html">Port Elizabeth 'To-Pay' and 'Paid'</a> handstamps including a photograph of the <a href="Port-Elizabeth-topay-paid.html\#portelizabethcover">earliest known cover</a> with the 'Paid' handstamp. 


The \href{../cape-of-good-hope/TPO}{TravellingPostOffice_Index.html} of the Cape of Good Hope are well represented here,  I having been fortunate enough to obtain the majority of items belonging to the collection of 
the late Mr. Naylor. 
Naylor together with Hagen wrote two books on the postal history 
of the Travelling post offices. Interspensed within these pages 
you can follow the story of the Cape of Good Hope Railways. From the first 
lines,such as the Cape-Wellington line, the 
<a href="TPO-Western.html">Western T.P.O.</a>, to 
items such as the <a href="TPO-Albany.html">Albany T.P.O</a> and 
the <a href="TPO-Zwartkops.html">Zwartkops Sorting Tender</a> 
including a scan of the only known cover from this T.P.O.

Some unusual Postal History material is included, 
such as the <a href="Exhibition-Kimberley.html">Kimberley Exhibition</a>
postmarks, with the only known cover of one of the 
caches dated July 1892 and a postcard posted by a member of an 
Austrian band from the exhibition! You might also be interested in the 
<a href="../ClanWilliam/Clanwilliam-private-stamp">Clanwilliam private stamp</a> 
which was used by the Clanwilliam postmaster during the Anglo-Boer war. 
The postal history of the <a href="ocean%20post%20office">Cape Ocean Post Office</a> 
is decribed in detail with a unique example of the Armadale Ocean Post Office Postmark.
            
            
Although stamps are not normally considered as part of Postal History collections the nature of the Cape of Good Hope stamps is such that I wouldn't have considered this work completed, if at least some mention was made of them. The triangular stamps are well represented with both the <a href="ADHESIVES/Perkins%20Bacon%20Introduction.html">Perkins Bacon</a> as well as the <a href="ADHESIVES/De%20La%20Rue%20Introduction.html">De La Rue Printings</a> and the <a href="ADHESIVES/Woodblocks%20stamps.html">woodblocks</a>. 

The rectangular stamps of the Cape of Good Hope present another aspect of the postal history of the Cape of Good Hope and have not been studied well. 
            
Some of the material illustrated here has not been published or exhibited before. The study of the '<a href="ADHESIVES/table-mountain.html">Table Mountain</a>' issue is one of them. Here Postal History material belonging to Mr. E.M.Sturman -the designer of the stamps- is shown. This stamp created a bit of controversy at it's time, in that it was the first stamp that did not have the 'Hope' shown. A few cuttings from newspapers regarding this controversy are shown as well as a letter from the Treasury awarding a bonus to Mr. E.M.Sturman for his design. The earlier issues including the <a href="ADHESIVES/%273%27%20on%20three%20overprint.html">'3' on Three overprint</a> with a study of <a href="ADHESIVES/%273%27%20on%20three%20overprint%20-%20varieties.html">errors and varieties</a>. No complete sheets for this issue have been recorded and a possible re-construction is shown <a href="ADHESIVES/%273%27%20on%20three%20overprint.html">here</a>. Other Cape of Good Hope stamps are listed separately as well as the rare <a href="Stamp-booklet-Cape-of-Good-Hope.html">stamp booklet</a> issued during the reign of King Edward VII. 

There are  a lot of topics covered in these pages but I am sure there are as much 
still not covered fully or discovered. I have endeavoured to go back and review most of the primary sources and major auctions and I would contributions in this respect
If you have any items in your collection that you think will 
benefit this project, please <a href="mailto:yannis_lazarides@yahoo.com">contact me</a> 
and I will do so, giving you full credit. If you want to become a co-editor for this country please contact us and we will be more than pleased to accommodate you.
           
          
                          