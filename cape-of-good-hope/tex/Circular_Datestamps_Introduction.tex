\chapter{Circular Postmarks}


\ph[width = .98\textwidth]{../cape-of-good-hope/circular-dot-at-bottom/P_O_ MOSHESFORD.jpg}{Ju 8, 1898 Postcard upgraded with \halfp posted at Moshesh's Ford in the Eastern Cape.}


The first circular datestamps of the Cape of Good Hope were issued in 1864. 
At first these were only issued to the general Post Office at Cape Town and 
the Port Elizabeth post office. Over a period of fifty years new and varied 
designs were issued. This section examines all the designs and variants 
both for the single circle as well as the double circle cancellations.

The postmarks are firstly classified broadly into single or double circle 
and then divided into various types.


\begin{enumerate}
\item The First Circular Dated Stamp of the Cape of Good Hope (1864)

\item The Second Circular Dated Stamp of the Cape of Good Hope (1864)

\item  The Circular Dated Stamps of the Cape of Good Hope (1869 to 1882)

\item The Circular Dated Stamp of the Cape of Good Hope - Dot at bottom-(1885)

\item  The Circular C.G.H. dated Stamps

\item      The Single Circle Dated Stamps of 1902 to 1903

\item  The Double Circle Dated Stamps of 1900 to 1902

\item  The Maltese Cross at Bottom Stamp of the Cape of Good Hope
 
\item  The Numeral at Bottom Stamp of the Cape of Good Hope

\item C.G.H. at bottom 

\item  'CAPE TOWN' at bottom

\item  General Post Office Dated Stamps of the Cape of Good Hope

\end{enumerate}

This section does does not deal with the more specialised cancellations used in the 
ancillary services such as parcel post, T.P.O., Ocean Post and others which rather belong to 
different sections of this collection.
 
It exclusively deals with circular markings used for dating and cancelling ordinary correspondence. 
The coverage of the earlier marks of this exhibit is mostly based on the James Perkins 
correspondence where material was adequate in my collection to form a coherent study with very little gaps. 
Now and then I have used colour prints of back of postcards and letters. 
This was intended to give the reader a flavour of the times when these postmarks were used.

 	    