\chapter{1893 One Penny Surcharge}
In March 1893 there were placed on sale the last of the provisional stamps 
issued by the Cape Colony, being the Two Pence value of the Cabled Anchor 
series overprint locally, "ONE PENNY"  the value obliterated by a bar. 

\ph[width = .50\textwidth]{../cape-of-good-hope/ADHESIVES/images/CAPE-One-Penny-Overprint-Brown-B4.jpg}
{1893 One Penny Surcharge}


Delays in new supplies of the 1d stamp forced the authorities to supplement 
the Treasury stocks of that value by the 
local conversion of 10,000 sheets of the Two Pence stamps into 
the lower value.


The surcharging was carried out by Messrs.  W. H. Richards \& Son in 
Capetown who in those days were the Government printers, 
in succession to Messrs. Saul Solomon \& Co.

The full sheet of 240 stamps was surcharged at each operation 
of the printing machine.

Upon  two  of the stamps on each sheet the stop after the 
surcharged words is omitted. Namely upon the sixth stamp on the seventh row of the upper left-hand pane, and the sixth stamp on the first row of the lower right-hand pane.{{footnote:1}}

 

\section{Double Surcharge}

\ph[width = .35\textwidth]{../cape-of-good-hope/clip_image002.jpg}{
1893 One Penny Surcharge
}


The stamp also exists with the surcharge doubly printed. It is believed that one whole sheet at 
least was sold at the Port Elizabeth Post Office in this condition, for all known used copies bear 
the Port Elizabeth postmark.

This must be one of the greatest undervalued rarities of the 
Cape of Good Hope. In my estimation there are probably less 
than 15 copies that have survived. There were probably about 10 
offered in auctions over the last 10-15 years, but I have as 
yet had to open an auction catalogue and haven't seen a woodblock on offer. 
Take your pick.

\section{Other Varieties}

There are several minor varieties in connection with this surcharge 
which are described below, and it is necessary to note that the 
bar of the surcharge cancelling the words "two pence" was set 
up separately in respect of each stamp upon the sheet, not in a continuous line.


\ph[width = .35\textwidth]{../cape-of-good-hope/clip_image007_0001.jpg}{
SAC 52 1d on Bistre Brown
}


Shades of the bistre-brown of the Two Pence stamps are to 
be found with the " one penny " surcharge.

 
\section{REFERENCE LIST}

Rectangular design of " Hope " sealed, with thin outer 
line of colour removed. Two pence bistre-brown. 
Watermarked Cabled Anchor. 
Surcharged by Messrs. W. A. Richards \& Son, of Capetown, 
with the words " one penny " in black, and with the original value, 
obliterated by a black bar.
"one penny" on 2d. Issued March 1893.
(Total number of stamps surcharged, officially recorded as 2,400,000.)
                            