\section{Unstamped Letters}

At first sight the cover shown in \ref{unstamped}, seems 
that the Postmaster is stating the obvious. He has applied 
a handstamp 'UNSTAMPED'. Well it should be obvious to anyone 
looking at the cover that it bears no stamps!

\ph[width = .80\textwidth]{../cape-of-good-hope/unstamped-cover.jpg}{
1867 Cover posted from Wellington, with oval datestamp in black, 
addressed to Somerset West, with Cape Town transit c.d.s., 
endorsed 'On Her Majesty's Service', (signed), 
with superb example of boxed "UNSTAMPED" in black (Goldblatt SP 17). 
Use of this handstamp is uncommon.\label{unstamped} 
}

Digging a bit deeper into the literature and postal rates of 
the Cape of Good Hope we find that in 1847, the new Postal 
Ordinance Section 10 provided that postmasters had to inform 
addressees of unpaid letters, which were
available to them at the post office on payment of 6d. ---the internal 
rate plus an additional half thereof. 
No special handstamps were provided for this purpose.

Pre-adhesive covers are seen with the rate of postage
inscribed in manuscript, in either red or black ink. 
Generally, red ink indicates that the postage has been paid 
and black is an indication that postage was to be collected 
from the addressee.

\section{Postage Stamps made Compulsory}

Prepayment of postage on letters was made compulsory 
by Act 21 of 1857, which further provided that where postage 
has not been paid at all, or underpaid, the postmaster-general 
was authorised to open such letters to determine the address 
of the sender. 

The sender then had to pay an amount of half the postage 
over and above the normal postage rate thereon. 

An internal letter of this period with a 6d. or 4d. and two 1d. 
postage stamps would indicate that this procedue has been followed.

Postmasters would also write "Unstamped" on letters, but 
special handstamps (SP17 and 18) were later provided for this purpose.

Effectively, the way I understand it this was the first 
attempt at a 'POSTAGE DUE' mark.      