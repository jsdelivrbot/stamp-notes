\chapter{Duplex Cancellation}

A duplex cancel was a hand stamp used to cancel postage stamps and imprint a dated postmark applied simultaneously with the one device. The device had a steel die, generally circular, which printed the location of the cancel, together with the time and date of cancel. This die was held in place by a handle with an obliteration marker, often oval shaped, off to the right side that was applied over the postage stamp to prevent its reuse. The ink came from an ink pad.

In the Cape of Good Hope, the first one was introduced in the latter part of 1864 in cape Town and reputedly also in Port Elizabeth. The device enabled postmasters to perform two operations --- defacing and dating --- simultaneously. The use of the Thick Bar Numeral Canceller appears to have been limited to these two large post offices, which also continued to use the single type of handstams. The 1864 duplex has an oval 21 by 28 mm, wheres the accompanying circular date stamp measures 20 by 28 mm, teh diameter of teh circle being 23 mm. Goldblatt denotes this as DBN 1. 

A later issue also of limited use is distinguishable, as the bars are slightly thinner and the oval measures 20 by 28 mm, teh diameter of the circle being 22 mm and this is denoted as DNB2.


In 1870, Thin Bar Duplex Numeral Cancellers (DNB 3 and 4) were allocated to Cape Town and Port Elizabeth. There are two types. the oval of DNB 3 measures 22 by 28 mm, with a circle of 23 mm diameter, and its numeral is thinner than that of DNB 4, where the oval measures 21 by 27 mm and the diameter of the circle 23.5 mm.

\ph[width = .90\textwidth]{../cape-of-good-hope/duplex-cancellation.jpg}{1875 Cape of Good Hope
1872 envelope to England bearing a 1/- cancelled by a barred numeral '1' CAPE TOWN MR 6 72 duplex canceller. On the reverse is a Chester AP 2 75 arrival cancellation. It would appear as though the Cape Town canceller had a date 'error' as the year is '72' and should have been '75'. 
£100.00
Reference: G2471  Goldblatt DNB 3. }

These were heavy to use and their usage did not last long. Examples are found with date setting errors as shown above.