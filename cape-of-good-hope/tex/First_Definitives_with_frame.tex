\chapter{Seated Hope} 

\section{The First Rectangular Stamps (First Definitive Issue 1864-1877)}

The first rectangular stamps of the Cape of Good Hope were issued 
in four denominations. The, one pence red, four pence blue, 
six pence violet and the one Shilling green. These were issued 
over a period of time as the stocks of the triangular stamps were 
being depleted. The new design had a 'Seated Hope' surrounded by a frame. 
Note on the illustrations here the outer frame. In later issues of the 
same design this frame was removed. This outer frame is the distinguish 
characteristic of this issue.

\ph[width = .30\textwidth]{../cape-of-good-hope/ADHESIVES/rectangulars-seated-hope/1d-noframe.jpg}{
Spiro Cape Triangular Forgery of the One Penny Red Stamps.}

\ph[width = .60\textwidth]{../cape-of-good-hope/7034_9_1.jpg}{
Auction: 7034 - The Rectangular Issues 
Lot: 9 Cape of Good Hope 1864-67 Issue with Outer Frame-Line Imperforate 
Plate Proofs 6d. pale lilac and 1/- green, both with clear or good to 
large margins and with part original gum; the 1/- with some creases though 
of good appearance. Photo Estimate \pound; 300-400 Sold for \pound750.
}




The first of the rectangular stamps of the Cape of Good Hope to make 
its appearance was the One Shilling green. This was placed on sale in 
the early part of January 1864.

This stamp should not have been issued until about a year later. 
Its early issue was caused by a change of postal rates,  is to be 
found in the fact of the new mail contract entered into between the 
British Government and the Union Steam Ship Company, early in 1863, 
for a monthly mail service to the Cape of Good Hope, under subsidy, 
towards which a share was contributed by the Cape Government.

\ph[width = .90\textwidth]{../cape-of-good-hope/ADHESIVES/rectangulars-seated-hope/Rectangular-Cancelled-One-Shilling.jpg}{ }

 
 

It is unnecessary to enter into the details of these new arrangements, 
which, were of a reciprocal nature, and which resulted as from the 
1st April, 1863, in an increased rate of postage being charged 
upon half-ounce letters between England and the Cape and vice 
versa, from 6d. to one shilling, if carried by the mail steamers, 
and a reduction to 4d the half ounce by private ships for the same service.

This, however, whilst interesting, is some-what irrelevant to the 
present purpose, which is to show the reason for the largely 
increased consumption of the one shilling stamps at the Cape of Good 
Hope about this period, and which was caused through the 
altered rate of charge.


The effect was that the Treasury stock of triangular One Shilling stamps
ran out at the end of 1863, and the first supply of the stamps of 
that denomination from the new die prepared by Messrs. De La Rue \& Co. 
which had been received in the Colony during the previous June, 
were called into requisition.


Small stocks of the One Shilling triangular stamps were, however, 
still on hand at various country post offices, and for some little 
time both shapes were in concurrent use.

A considerable number of different printings of the One Shilling 
stamps were made between 1863 and 1879, all upon the paper 
watermarked Crown C-C, and they can be graduated in a range of 
shades from pale yellow-green to the rare deep blue-green. As, 
however green  was the official colour description throughout, it 
is impossible to allocate any particular shade, however distinctive, 
to any particular period from the records.

The total number of these stamps forwarded to the Gape between 
1863 and 1879 was 2,548,080.


Of this quantity approximately 410,000 were supplied to 
the Griqualand West Administration for postal purposes in that 
territory, and were overprinted with the capital letter "G" in various types.


\ph[width = .60\textwidth]{../cape-of-good-hope/7034_6_1.jpg}{ 

Auction: 7034 - The Rectangular Issues 
Lot: 6 Cape of Good Hope 1864-67 Issue with Outer Frame-Line 
Die Proofs 1/- in green on glazed card (64x58mm.). Rare and most attractive. 
Photo Estimate \pound 1,500-2,000 provenance: Christie's Robson Lowe, 
1986 Sold for \pound3,500
}
<p class="small">The Rectangulars</p><div style="clear:both"></div>


 
\section{The 6d Stamp}
(Issued prior to 21st March, 1864)

This was the second stamp to be issued in this series. The first supply
was forwarded to the Cape on the 5th May 1863, with the first consignment
of the One Shilling value.

\ph[width = .60\textwidth]{../cape-of-good-hope/7034_11_1.jpg}{
Auction: 7034 - The Rectangular Issues 
Lot: 11 Cape of Good Hope 1864-67 Issue with Outer Frame-Line Issued 
Stamps 6d. pale lilac block of four with large part original gum, 
fresh delicate colour; the upper left stamp with inclusion and a 
couple of small marks at top, the lower pair with trace of a bend 
though of fine appearance. A rare multiple. S.G. 25. Photo Estimate 
\pound; 250-300 Sold for \pound;350
}
<p class="small">The Rectangulars</p><div style="clear:both"></div>

As in the case of the One Shilling stamps a good many separate 
printing were made of the Six Pence upon the crown CC paper 
and are found in a range of colour tones ranging from lilacs to mauves.


\section{The One Penny Value}
(Issued May 1865)
The first order of these stamps was despatched to the Cape of 
Good Hope on the 19th November 1864. A request was made to 
have the order expedited as soon as possible as it was 
anticipated that they would be largely used for the requirements 
of the new Stamp Act passed at the previous session of the 
Cape Parliament. This order was shipped in two parcels and was 
received at the Cape in April and may 1865. It is presumed that 
they were issued in May 1865 the catalogues normally listing 
this as 1st of May.


Several printings were made of the One Penny value upon the Crown CC 
paper between 1865 and 1870 and a splendid range graduating from 
pale rose-red to rare intense carmine.

\section{The Four Pence Value}
(Issued August 1865)


 
Allis records that the Four Pence stamp of the rectangular shape 
was issued in August 1865.

\ph[width = .70\textwidth]{../cape-of-good-hope/7034_5_1.jpg}{
Auction: 7034 - The Rectangular Issues 
Lot: 5 Cape of Good Hope 1864-67 Issue with Outer Frame-Line 
Die Proofs 4d. in deep blue on wove paper (26x32mm.) 
affixed to card (65x58mm.). Rare and most attractive. Photo 
Estimate 
\pound 1,500-2,000 provenance: Christie's Robson Lowe, 
1986 Sold for \pound3,500
}
<p class="small">The Rectangulars</p><div style="clear:both"></div>


Between 1865 and 1871 several further printings were 
made all of which were in lighter shades of blue. In 1871, 
however, the Postmaster General in requesting further supplies 
drew attention to the desirability of the stamps of this value 
being in a deeper shade of blue. He suggested a colour similar, 
to the colour of the English 2d stamp and stated as the reason 
for his request, that by gas-light the pale blue assumed a 
greenish colour and was mistaken for the green of the 
One Shilling value stamp.

 

A small number of these stamps were supplied to the Griqualand West 
Administration where they were overprinted with a "G".
Colours normally listed in catalogues are as follows:

Blue (shades)
Ultramarine
Blue (1872)

This stamp is sometimes found with the outer frame-line 
missing on the side or sides and or top or bottom. This effect 
was due to the wearing of the plate and resulted in the outer line 
being discarded when a new plate became necessary.

The issues that follow bear this characteristic (see Second Definitive Issue).

 
\ph[width = .70\textwidth]{../cape-of-good-hope/7034_7_1.jpg}{
Auction: 7034 - The Rectangular Issues 
Lot: 7 Cape of Good Hope 1864-67 1d. carmine-red and 4d. 
pale blue, both handstamped "specimen" (D5) diagonally, 
both without gum; the 1d. with a few small imperfections. 
Rare. Photo Estimate \pound 350-400 Sold for \pound320
}
<p class="small">The Rectangulars</p><div style="clear:both"></div>

 
\section{Specimens From the De La Rue Records}

\ph[width = .99\textwidth]{../cape-of-good-hope/7034_8_1.jpg}{ 
Auction: 7034 - The Rectangular Issues 
Lot: 8 Cape of Good Hope 1864-67 Issue with Outer Frame-Line Specimen 
Stamps 6d. pale lilac and 1/- green, both with manuscript "Specimen" 
applied diagonally and affixed to piece (179x51mm.) from the De La 
Rue record book marked "New plate April 1863" and giving details o
f the sheet size and watermark. A unique pair. Photo Estimate \pound600-800 
Sold for \pound800.
}
<p class="small">The Rectangulars</p><div style="clear:both"></div>



                                                              