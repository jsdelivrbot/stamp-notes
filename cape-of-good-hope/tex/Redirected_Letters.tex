\chapter{Redirected Letters}

\ph[width = .70\textwidth]{../cape-of-good-hope/redirected/Redirected-Entire.jpg}{
Redirected Letters}


\ph[width = .70\textwidth]{../cape-of-good-hope/redirected/Kimberley.jpg}{
1906 Incoming picture postcard from UK bearing GB KE VII 1d tied 
by London (2 June) cancellation. Addressed to Toise River 
(21 June cds) where it was redirected to Kimberley (23 June). 
Again re-directed to Vryburg. It received a fine strike of the 
"OFFICIALLY REDIRECTED KIMBERLEY (1)" 
boxed cachet in violet. With manuscript 1/2d charge in black. 
(Unrecorded by Goldblatt).
}

\ph[width = .70\textwidth]{../cape-of-good-hope/redirected/redirected-president-steyn.jpg}{
1905 Redirected letter posted in Holland and addressed to Ex-President Steyn, 
in Cape Town and redirected to the Onzerust Post Office at the  
Orange River Colony.  The cover had the 
"OFFICIALLY REDIRECTED"(Goldblatt SP 36) handstamp, applied at the front. 
Back stamped with double circle Orange River Colony  
"POSTE RESTANTE" dated 8.03.05. No fee was charged for redirection.
Martinus  Theunis Steyn 1857-1916 was the sixth and last 
president of the independent  Orange Free State 1896 to 1902. 
He fled the country during the Anglo-Boer War 
and returned to South Africa on the 12 February 1905. 
On 21 February 1905, after having spent 9 days in Cape Town, 
President Steyn and family, were welcomed with tremendous 
rejoicing by all his old war companions and friends who 
awaited his arrival on Kaalspruit station, the station just 
before Bloemfontein, and drove him to his farm 'Onzerust' 
in the Orange Free State, 3 miles away.  He probably received 
the letter on his arrival.
}
 

Letters required to be redirected because the addressee had 
moved needed to be additionally stamped with an amount equal 
to the original postage; an amount that was required each time 
the letter was redirected. If the addressee refused to pay 
the additional redirection fee, the letter was returned to 
the sender, who was then liable for double the postage thereon.

From 1890 no charge was made for redirection, provided the letter was unopened and reposted within a short time. The post office used a rubber handstamp (SP 36) when instructions to redirect were received.

Boxed rubber handstamps (SP 37 and 38) were brought into use at Cape Town and Port Elizabeth. 
These have numerals after the name of the town; 1, 2 and 3 being used in Cape Town and 1 and 2 at Port Elizabeth. A similar handstamp (SP 39) was issued to Wellington, but this does not show a numeral. These handstamps were in use from about 1904.
Kimberley redirected letter

1906 Incoming picture postcard from UK bearing 
GB KE VII 1d tied by London (2 June) cancellation. 
Addressed to Toise River (21 June cds) where it was 
redirected to Kimberley (23 June). Again re-directed to Vryburg. 
It received a fine strike of the
\textsc{OFFICIALLY REDIRECTED KIMBERLEY (1)}
boxed cachet in violet. With manuscript 
1/2 d charge in black. (Unrecorded by Goldblatt).

There were also single- and double-lined redirectional 
handstamps (SP 40 and 41) 
in use at King Williamstown in 1906 and at Wynberg in 1909.
                      