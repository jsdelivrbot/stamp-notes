\chapter{1882 Twopence Bistre}  
\subsection{Issued July 1882}

The new Twopence stamp was required to meet the needs arising from the Postal Consolidation Bill which had recently been passed by the Parliament.

In April 1882 the Postmaster-General ordered the preparation of a new plate for
the 2d value to coincide with a proposed rate alteration. The Colony requested that the colour should be of 'a dark amber brown, provided that a shade can be obtained which will not be capable to be mistaken for the reds and maroons at present used'<sup>1</sup>. The requirement channelled through the Crown Agents was for a quarterly supply of 960,000 and the first supplies were to be delivered to the Colony by July 1st.

The new stamp was issued to accommodate a reduction of inland postal rates for half-ounce letters from 3d to 2d, to be effective 1 July 1882. A bistre-brown would be used as previously recommended for the 3d value.

\ph[width = .50\textwidth]{../cape-of-good-hope/adhesives/2d-bistre-proof.jpg}{1882: Two Pence die proof on glazed card.}

\ph[width = .89\textwidth]{../cape-of-good-hope/82-83-die-proof.jpg}{
882-83 DIE PROOF for 2d (SG 42) in black on glazed card marked BEFORE HARDENING and 
dated 12 JUN 82. VF, scarce. ...ebph/019	
Price: \$ 510
}



De La Rue wrote to the Crown Agents on May 17th:

\begin{quotation}
With reference to your letter of this afternoon, covering a copy of a letter received from the Cape of Good Hope, respecting a supply of 2d. Postage stamps and a \halfd. Newspaper Wrappers, we beg to say that the instructions are fully understood by us, and that we will immediately take steps to prepare the necessary Postage die after the pattern of the other dies.

In regards to the new Newspaper Wrappers die we would strongly recommend that this should be a different design to the 1d. die, so that distinction between duties may not depend upon colour alone.

We therefore, herewith enclose a design which we have prepared for such a die, attached to which a specimen of the colour we should propose to print the Wrapper in.

We regret to say that with the utmost exertion it would be impossible for us to get a consignment of the new stamps and Wrappers ready until about the middle of July next, so that it occurred to us that you might wish to have some temporary 2d stamps prepared by overprinting in the manner by shown upon the enclosed specimen the stamps printed in burnt umber as directed in the Cape of Good Hope letter and overprinted in black. 
\end{quotation}


On May 23rd the Crown Agents informed De La Rue that their proposal for an overprint was not accepted and that they should hasten the production, with a view to shipment not later than the middle of July.

The die and plate of 240, multiples, with 600,000 stamps, were invoiced two days later.

The Stamps were overprinted in 1883 with  the \href{../cape-of-good-hope/1893_One_penny_surcharge}{One-Penny surcharge.}



\subsubsection{Production Timeline}

\begin{description}
\item[25 April 1882] Cape letter to Crown Agents ordering the production of stamps.

\item[17 May 1882] De La Rue letter to Crown Agents proposing overprinting.

\item[23 May 1882] Crown Agents approve production of stamps.

\item[25 May 1882] Die plate and 600,000 stamps invoiced.

\item [12 June 1882] Before Hardening die proof.
\end{description}














1. Easton pg356-57. 
1. Eston pg722

              