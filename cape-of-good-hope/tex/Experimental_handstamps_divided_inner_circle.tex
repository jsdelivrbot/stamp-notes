\section{The Divided Inner Circle Datestamp of 1898}

\begin{marginfigure}
\centering
\includegraphics[width=.80\textwidth]{../cape-of-good-hope/Divided-Inner-Circle/Postmarks.jpg}
\caption{ }
\end{marginfigure}  	 


Contrary to other Postal History studies (Goldblatt and others 
do not treat it as an experimental handstamp), 
I have treated this datestamp as the last of the experimental 
handstamps due to the fact that 
this followed directly the use of the Compass Wheel. 
It was also the first datestamp to use the actual time, instead 
of a control letter. 
The experimental handstamps were all superceded later by the 
double circle datestamps of 1900-1902 
that incorporated most if not all the features that the post 
office required, such as the full time 
shown as am or pm, a design that was not subject to wear - a 
circle is obviously the best and a double 
circle even better.



Goldblatt does not indicate the date of issue for the handstamp.
I could not find exact records as to when in 1898 this datestamp was 
first used but from records 
in this study it appears to be August 1898.


The design consists of an outer circle with a diameter of 39 mm and 
two semi-circular lines, separated by the date and time control. 
The words "G.P.O. Cape Town" appear at the top, and "Cape Colony" 
at the bottom of the handstamp (DDS 1 to 3). 


Two variant types may 
be distinguished. The first (DDS 2) has the semi-circular lines 
connected by baselines to form two segments, while the other (DDS 3) 
shows the baseline only partially extended. When observing these 
designs Goldblatt notes that ..."the immediate reaction is to 
attribute these baseline deficiencies to wear, but from many examples 
seen the variants seem to be constant". 

\ph[width = .70\textwidth]{../cape-of-good-hope/Divided-Inner-Circle/EARLIEST-COVER-AUGUST-1898.jpg}{Cover dated August 1898 }
The General Post Office in Cape Town began using this new type of 
datestamp in 1898 for both backstamping of letters as well as 
defacing of stamps. It appears that The Divided Inner Circle 
Datestamp was used exclusively at the G.P.O. head office of 
the Cape of Good Hope.

\ph[width = .70\textwidth]{../cape-of-good-hope/Divided-Inner-Circle/Earliest-Cover.jpg}{Earliest Known Cover.}

I am in agreement with Goldblatt that from the examples of all 
three can be observed. However, I am not certain that all three 
datestamps were different designs. It can be observed from the 
examples shown in this page, which were during the first month
of operation of this datestamp, the first issue was that of DDS 2, 
incorporating full lines into the design.

It is possible that as the datestamp worn out - especially given 
the fact that during the Anglo-Boer war there was an increase in 
correspondence - the datestamp was filed out to make the insertion 
of the time and date control easier.

The usage of the datestamp was terminated late 1901 or early 1902 
superseded by the Double Circle Datestamp. This is not surprising 
as the impressions made by this datestamp are normally very poor. 
It is also very difficult to read the date. On a score of 1-10 from 
all the experimental hand-stamps this one gets a 3. 



\ph[width = .90\textwidth]{../cape-of-good-hope/tochina.jpg}{
CAPE COLONY 5/12 98 Ganzsachenkarte 1 p dt to the battleship 
SMS "Kaiserin Augusta" in Kiautschou. With vs. * A passage mark 
TSINTAU China 19/2 99, along with zahlr. other stamps to Hong Kong 
and Manila continues sent. SHANGHAI CUSTOMS including Rs 18/2 99th
Automatically generated translation: CAPE COLONY 5/12 98 postal 
stationery postcard 1 p to the German. Warship SMS "Empress Augusta" 
in Kiautschou. With one face on a route passing cancellation TSINTAU * 
China 19/2 99, together with numerous further postmarks to Hong Kong 
and Manila expanse envoy. At back. Amongst other things CUSTOMS 
Shanghai 18/2 99th
Auction
Minimum Bid:
650.00\thinspace\euro 
WuttenburgspaceSept 2012
}








 

 

                                       