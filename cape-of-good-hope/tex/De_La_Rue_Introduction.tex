
\subsection{The De La Rue Printing
} 


It is recorded in official records that at the end of January 1862 the dies, plates, watermark paper moulds, e.t.c. were handed over to the Crown Agents by Messrs. Perkins Bacon \& Co..

These were transferred to the new Contractors-Messrs De La Rue \& Co.- in May 1862. This offered an advantage both in cost and in security.

An order was issued to the new Contractor and on the 5th February, 1863, the six Pence and One Shilling stamps were despatched to the Cape of Good Hope authorities. They arrived at Cape Town on the 14th of March while the Four pence was forwarded in May.

\ph[width = .80\textwidth]{../cape-of-good-hope/ADHESIVES/triangulars/penny-block.jpg}{
}

In April 1863 another order was placed for the One-Penny and the Four Pence. And despatched to the Colony in December of the same year.

The last printing of the One-Penny was despatched by the Cambrian on the 5th March and the Four Pence was despatched on the s.s. Athenson the 5th May 1864
The stamps were printed on watermark Anchor paper.

\subsection{How to Distinguish Between the Perkins Bacon and the De La Rue Issues
} 
The stamps can be distinguished from the Perkins Bacon Issues by a printing that is not so distinct -as the plates started to get worn- as well as different printing shades. The stamps are described as having a wooly or flaky appearance. This implies that the engraving lines are blurred and in extreme cases the background is a uniform mass of colour. The flaky appearance shows the ink lying in tiny solid areas, as if it had not penetrated a well-chalked paper surface.


\ph[width = .60\textwidth]{../cape-of-good-hope/ADHESIVES/triangulars/differences between de la rue and bacon printings.jpg}{ }

 
Unterschied der Drucke der Kap-Dreiecke von De la Rue und Perkins Bacon
At left is the Perkins Bacon Issue at Right is the De la Rue issue.the most pronounced difference is on the hand and I normally use that feature to distinguish the stamps.

\phl[width = .50\textwidth]{../cape-of-good-hope/ADHESIVES/triangulars/Four-Pence.jpg}{ }


\subsection{Specimens
} 
\ph[width = .50\textwidth]{../cape-of-good-hope/ADHESIVES/triangulars/1 shilling specimen file.jpg}{ }






You can check this one again below
Four Pence pair De la Rue printing
A Four Pence pair. Note the distinct trumpet shape separation between the two stamps. The top is Die A and the bottom is Die B



\subsection{One Shilling Emerald Green}

From p8 Philatelic Record 1898 VOL. XX. JANUARY TO DECE
1898.

Cape  Emerald Shilling 5s Find
By MAJOR EVANS.

\begin{quotation}
From "The Monthly Journal." The One Shilling triangular emerald-green of the Cape of Good Hope
has always been a rather scarce stamp, Unused in mint condition,
especially in unsevered pairs or blocks, they are decidedly difficult stamps to find; and this will be easily understood, when we re- member that the stamps in this colour belong to the comparatively small supply of that value printed by Messrs. De La Rue and Co., and sent out
in 1863, not long before the triangular stamps were superseded by the less striking but far more convenient rectangular. To this fact, and also no doubt
to the fact that in those days the great majority of collectors were content
with a single copy, and preferred that copy obliterated, we owe it that the
emerald-green Shilling, in fine unused condition, has not come down to us in such abundance as we could wish. 


One hundred and fifty-eight sheets is the total number stated in the London
Society's Africa book to have been despatched to the Colony in January, 1863 ; this was the last lot of triangular Shillings that was supplied, and the only lot printed in emerald-green. It was, therefore, rather a shock to me when,
a few days ago, a friend who had asked me to meet him, as he had something curious to show me, pulled a modest-looking roll out of his pocket, and
carefully unwrapped one of these very 158 sheets, almost entire, lacking only two specimens out of the 240 ! It was a real find, having turned up in a mass
of old papers, supposed to be of very little value, but containing a certain number of curiosities, amongst which this will certainly take the highest place. 

The sheet is in excellent condition, perfectly clean, with original gum and
margins, except where a single pair has been cut out, quite complete; and
one can only wonder what led its original owner to stow away nearly \pound12
worth of stamps (face value) in this manner, for there can be little doubt that
it has never before been in the hands of a stamp collector, and it would appear
to have been accidentally preserved, as no other stamps of so early a date have
yet been found in the pile. 

The impression is not even in tint throughout, and it seems evident that
either one end of the plate was somewhat worn, or that end was not so
heavily inked as the other when this impression was printed, one end of the
sheet being distinctly paler in shade, and showing a slight want of ink at some
of the points where the corners of four stamps meet. The other part of the
sheet is of a beautiful deep colour.

It is, of course, on the usual Anchor-watermarked paper, and as sheets of these
triangular stamps are not often to be met with nowadays, I thought it of interest to take some notes of the dimensions of this one, and of the arrangement of the watermarks, \&c. The arrangement of the stamps is well known; the sheet of each value contained 240 copies, in fifteen rows of eight pairs in each
row. The watermarks are so placed that an Anchor should appear upon
each stamp, with its stock pointing towards the top corner of the design. Surrounding the portion of the sheet intended to receive the impression of the
plate, is a frame of five parallel lines in watermark, interrupted twice along each of the longer, and once in the centre of each of the shorter sides by theNewfoundland Provisional ic. word " POSTAGE " in outline capitals. The letters occupy the width of four of the lines, and the inner line is continuous all round the pane of Anchors. 

The size of the pane of stamps, in the case I am describing, is 263 x 500 mm., and of the sheet of paper (the watermarked frame extending quite to the edges) 282 x 536 mm. From these particulars it will be seen that it would be quite impossible to put a sheet into the press sideways, which it is stated in the Society's Africa book is the cause of parts of two Anchors sometimes appearing at the bottom of a stamp, instead of one Anchor in the middle. This misplacement of the watermark is evidently due to the sheet being put
into the press with the wrong side of the paper uppermost, which may frequently have taken place ; in this case the diagonal space between the two stamps of each pair would fall along, or parallel to, the stocks of the two Anchors, instead of between the latter, and so a portion of each Anchor would come at the bottom of each stamp. We thus see that this paper, which had the advantage
of having neither top nor bottom, but was always the right way up in one sense
of the term, possessed the disadvantage of having a right and a wrong side. 

Further examination of the sheet seems to show that the roller with which
the plate was constructed probably bore two impressions of the original die, carefully placed with the bases of the triangles parallel to one another, as the diagonal spaces are very regular, whilst the horizontal and vertical spaces between the rows of pairs are not so; the former, that is the spaces between
the rows of eight, vary from 2mm to 3 mm., and the latter, the spaces between
the rows of fifteen, varying also, but not to so great an extent, and hardly in any part exceeding 2 mm.

All these little details may appear to be of minor importance, but it is as
well to place information of this kind on record, as it is not always obtainable.
\end{quotation}









               