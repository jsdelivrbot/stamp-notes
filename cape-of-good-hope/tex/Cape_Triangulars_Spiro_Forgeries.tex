\chapter{Triangulars - Forgeries}

\subsection{Spiro Brothers Forgeries} 

The Spiro brothers made large numbers of forgeries, including a large 
number of Cape Triangulars from 1864 to about 1880. They had their 
own lithographic printing firm in Hamburg (Germany).

\ph[width = .70\textwidth]{../cape-of-good-hope/ADHESIVES/triangulars/SpiroCapeTriangularForgery.jpg}{
Spiro Cape Triangular Forgery of the One Penny Red Stamps.}

Strictly speaking the Spiro forgeries are fascimiles and were sold 
as such. A legitimate activity in those days. Over the years about 
500 different forgeries were made. These forgeries must be the most 
numerous of all stamp forgeries and can be found in almost every old 
collection.

The Cape Triangular forgeries were normally printed in sheets of 30 
as in the example shown here. In sheets they are difficult to find. 
As they look very attractive grub one if you can find it.

The stamps were never gummed, but can be found mostly cancelled. 
The typical 'Spiro cancels' are often the easiest way to recognize these forgeries.

Facsimiles were openly collected and from 1864 to 1875 the Spiro Brothers firm made
its living by meeting the need fully and openly.

Other early dealers also offered facsimiles as well as outright fakes or reprints.  
From George Hussey and Ferdinand Elb, to S. Allen Taylor and J. Walter Scott, 
livings were made by providing facsimiles and forgeries in addition to 
legitimate and bogus issues

                         