\section{The Compass Wheel Datestamp of 1891}

In the course of 1891 an an entirely new type of handstamp was issued. 
This stamp was only issued to Outdtshoorn and Cape Town.

\ph[width = .90\textwidth]{../cape-of-good-hope/square-circle/compass-wheel/cover-cw-01.jpg}{
}

The design has been described as a scalloped octagon,composed of 
32 small segments with a larger pentagon at each of the eight points. 
The segments encompass a circle of 16 mm, containing the control 
letter and the day, month and year set out in three lines.

I have found no evidence of use of this datestamp in 1891. 
The only evidence I could find was from 1892. So unless this stamp was issued to 
Oudtshoorn earlier than Cape Town, it only came into use in 1892.
  	 
\ph[width = .35\textwidth]{../cape-of-good-hope/square-circle/compass-wheel/cw-1.jpg}{ }

\ph[width = .35\textwidth]{../cape-of-good-hope/square-circle/compass-wheel/cw-1.jpg}{ }

The Compass Wheel datestamp of Cape Town (CW 1) differs slightly from 
that of Outdtshoorn (CW 2). The former adds "Cape Colony" to 
the town name and the latter has the letters C.G.H. These were 
the penultimate of the experimental handstamps commissioned as 
trials for a combined obliterator and datestamp.

 

                 