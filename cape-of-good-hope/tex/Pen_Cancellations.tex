\section{Pen Cancellations}

A number of the early Cape of Good Hope stamps were legitimately 
cancelled by pen on post office instructions as no cancellers 
were provided. These \textit{pen cancellations} or defacement of stamps was
intended by the postal authorites to prevent their re-use and 
form an interesting branch of Cape of Good Hope Postal History. 
Care should be taken not to confuse these pen cancelled stamps with 
those pen cancellations for fiscal purposes.


\ph[width = .90\textwidth]{../cape-of-good-hope/pen-cancellations/troe-troe.png}{
Troe-Troe Manuscript cancel wrapper  endorsed "In H M Dienst" 
but franked on reverse strip of 3x4d (SG 24a) pen  cancelled and with 
manuscript 
"Troe Troe" 21/11/64" alongside. Cape Town arrival 
b/s "No 25 64";. Wrapper opened out  for display.  
}  
   
These post offices were in the majority agencies generally operated by 
storekeepers, farmer's wives or missions and were often remote - 
especially in the Karoo areas where the quantity of mail was limited. 
The functions of postal agents were to receive letters for onward 
transmission, or hand letters to addressees. They also sold postage 
stamps, which they could purchase at a small discount, but many 
postal agencies were not provided with a cancelling device.

Thus, when an agent was handed a letter bearing an unused adhesive, 
he would occassionally apply a pen cancellation, sometimes inscribing 
the name of the postal agency on the stamp for this purpose. These 
covers are not common.

\section{Pen Cancellations for Fiscal Purposes}

The pen cancellation of revenue stamps has always been acceptable 
and this defacement for fiscal use is still practiced to-day. 
The defacement of stamps for fiscal purposes was also carried out 
by means of handstamps. This was carried out according to the 
Stamp Act of 1864. Most bank handstamps incorporate the word bank. 
These are described in detail in Goldblatt. These sometimes are 
difficult to recognise and the collector should be careful in buying 
pen cancelled stamps. These should preferably be bought either with 
a certificate or on cover.

\ph[width = .98\textwidth]{../cape-of-good-hope/14018_420_1.jpg}{Auction: 14018 - The Anglo-Boer War, 1899-1902, featuring Occupation and Siege Issues, The Harry Birkhead Collection 
Lot: 420 (x) Zeerust
The town was occupied on 5 June 1900 by Rhodesian Troops under Lt. Col. Plumer where he joined forces with Baden-Powells troops from Mafeking and Otto'shoop. The force then moved forward to occupy Brackfontein on 10 June and Rustenburg on 14 June
Postal History
1900 (15 July) envelope registered "No 6" from Otto's Hoop to London, bearing Cape 1d. carmine (5) sharing manuscript "Otto's Hoop 15-7-1900", London Registered oval d.s. (11.8) alongside, the reverse with Mafeking (16.7), Cape Town (20.7) and London (11.8) datestamps. A most unusual cover after the Boers retreated leaving neither stamps nor cancellers. Photo 

Note: Otto's Hoop, also known as Malmani Goldfields, was occupied on 2 June 1900. Sold for £1,200}

      
 


                            