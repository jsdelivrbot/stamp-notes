\section{The Fifth Definitive Issue}




\section{Water mark cabled anchor}

1896-98


ewthought{In the Government Gazette} of the 28th December 1894, it was announced 
that arrangements had been made for the entry of the Cape Colony into the Universal Postal Union as from the 1st January 1895. This involved the adoption of the Postal Union colours of green, red and blue for the halfpenny, One Penny and Two pence-halfpenny stamps respectively. This necessitated a great amount of change, which was implemented, gradually between May 1896 and September 1898.
All the stamps of this series were printed from the former plates 
at the works of Messrs. De La Rue \& Co., upon the Cape Cabled Anchor paper, 
perforated and gummed as before.


\ph[width = .30\textwidth]{../cape-of-good-hope/ADHESIVES/rectangulars-seated-hope/watermark-Cabled-Anchor.jpg}{
Original Die Showing the Outer Frame Line (You can see the frame easier if you look at the bottom of the stamp)
}

\section{The Two Pence Halfpenny}

(Issued May 1896)

This was issued in ultramarine and several printings were made so 
that pronounced shades are obtainable.

\ph[width = .30\textwidth]{../cape-of-good-hope/ADHESIVES/rectangulars-seated-hope/Two-and-Half-Penny.jpg}{
Original Die Showing the Outer Frame Line (You can see the frame easier if you look at the bottom of the stamp)
}


\subsection{The One Shilling}

(Issued May 1896)

This was issued in yellow ochre. Although this stamp was printed previously in green 
the alteration was necessary in view that green has now been appropriated for future supplies 
of the Halfpenny stamps.

\subsection{The Five Shilling}

(Issued June 1896)

In the case of the new Five Shillings stamps the change is one of shade only. All previous stamps of this value had been in orange-yellow and the newcomer was printed in a deeper tone, with a tinge of brown in the composition

 
\subsection{The Half Penny}

(Issued December 1896)

The Halfpenny stamp made its debut in its new Universal Postal Union colour-green which had in the meantime being discarded by the One Shilling value in its favour.
Numerous printings occurred and a great variety of shades are to be found.                