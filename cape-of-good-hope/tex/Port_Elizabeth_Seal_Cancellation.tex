\chapter{Port Elizabeth Seal Cancellation}    

This cancellation was first  mentioned by D. Alan Stevenson in his book
\textit{The Triangular Stamps of Cape of Good Hope} 

\ph[width = .60\textwidth]{../cape-of-good-hope/8263.jpg}{1031 South Africa: Cape of Good Hope: 1863-64 4d. blue used with Port Elizabeth 'seal' cancellation. The words "Port Elizabeth" are clearly discernible.
Estimate: \pound80--\pound100, Grosvenor June 2013.
 }   
 
Stevenson noted:
Seal Defacer - several specimens are known of Triangular stamps defaced in black by what appears to be the reverse of a seal of a mail bag. Details are not clear.  Inside two circles of 25 and 15mm diameter, appear the words  \textsc{post office} or \textsc{official}. The inked centre is divided by two straight lines 3.5mm apart. It is not certain whether there are other words between the two cirles in the inked half circles or between the two lines. The mark is seen cancelling a Perkins Bacon 4d stamp on letter from Port Elizabeth dated 28 January 1859 and addressed to Cradock. Possibly it might have been used on letters handed in as the post bag was being sealed or thereafter.'

Later Goldblatt, published a note in the London Philatelist<sup>2</sup> in which he described, how he used elements of the handstamp visible from several specimens that came up for auction in major collections to reconstruct the wording and the design of the handstamp. He later included this information in his book<sup>3</sup>.

\ph[width = .70\textwidth]{../cape-of-good-hope/port-elizabeth.jpg}{The handstamp as reconstructed by Goldblatt, Figure 1 represents the combination of the visible elements from the illustrations of the auction catalogues, whereas Figure 2 illustrates a projected impression of this Strike. }

\ph[width = .90\textwidth]{../cape-of-good-hope/port-elizabeth-seal.jpg}{ }
\ph[width = .90\textwidth]{../cape-of-good-hope/huston.jpg}{From 'The C. Emerson Huston Sale'. }

Probably about ten copies of the seal have survived. 

\subsubsection{References}

2 \LP{Goldblatt R.}{Cape of Good Hope, The Port Elizabeth 'Seal Defacer' 1859}{July-August 1983}{92:1087-88, p.96 (LP1087.pdf).}    

3 \goldblatt{78}        