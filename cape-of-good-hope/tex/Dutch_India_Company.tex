\chapter{The Dutch East India Company}

The Dutch East India Company was established on 20 March 1602. 
Its charter granted by the authorities of the United Netherlands 
conferred a monopoly to trade east of the Cape of Good Hope and 
west of the straits of Magellan, effectively granting it the rights to the 
Pacific and Indian Oceans.

\ph[width = .95\textwidth]{../cape-of-good-hope/history/smallpox_cape_files/khoisan1706.jpeg}{
}


 


The Company's board of directors was known as the Council of Seventeen. 
The chamber of Amsterdam having provided half the capital 
was represented by eight members; Zeeland by four;Delft, 
Rotterdam, Hoorn and Enkhuizen by one each and the seventeenth 
member was nominated by the last five.

The Council of India was headed by a governor-general who controlled 
the Company's foreign posts, with headquarters on the island of Java. 
These controlling bodies were separated by a sea journey of seven months, 
and thu the council in many instances was obiged to make decisions of 
which its superiors, the 'Seventeen', were informedonly retrospectively.

Each of the company's outpotst was managed by a Council of Policy. 
This was the highest local authority, which operated under the 
chairmanship of the local commander or governor. Second to him 
was the aptly named 'Secunde'. Law was administered by a judicial 
body called the 'Council of Justice'. In the early days of the 
Cape of Good Hope, this was the same body as the Council of Policy.

The 'Fiskaal' was the Company's prosecuting official, responsible 
directly to the 'Council of Seventeen' and not to local officialdom. 
This position was abused and the 'council of Seventeen' appointed an 
'Independednt Fiskaal', who was entiled to a third of the fines , 
thus supplementing his salary. This post was abolished in 1793 and 
that of fiskaal re-instated, but now the fiskaal was directly 
responsible to the governor.

On 1647 on e of the finest V.O.C.ships the Nieuw Haarlem was 
wrecked on the shores of what is now Table Bay. All aboard managed 
to survive and most of the cargo was saved. The crew was obliged to 
grow vegetables and by bartering cattle with the local inhabitants. 
They were rescued after five months in March 1648 by the homebound fleet. 
By coincedence a young merchant Jan Anthoniszoon van Riebeeck was also 
returning on the Conick van Polen. On their return a report was made 
recommending the Cape of Good Hope as a victualling station. 
After careful consideration Jan van Riebeeck was appointed 
as commander of a settlement which he had to establish at the 
Cape of Good Hope on behalf of the Dutch East India Company. 
Jan van Riebeeck arrived at Table Bay on 6 April 1652 in his 
flagship the Drommedaris, accompanied by the flute Reijger and the 
'jacht' Goede Hoope and followed by the Walvis and the Oliphant.

Hisinstructions were to build a fort for the protection of the 
settlement, establish a farm to grow vegetables and to secure 
pasturage for cattle. He also had to establish a hospital.

Read further on the Dutch Settlements in the Cape of Good Hope.

In 1794, the Dutch East India Company went bankrupt and in 1795 
the English seized the colony, the Dutch  surrender in 1795. 
This is known as the capitulation of Rustenburg.

     