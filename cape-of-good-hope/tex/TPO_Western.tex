\section{Cape of Good Hope - Postal History } \subsection{Travelling Post offices of The Cape of Good Hope} \subsection{Western T.P.O.} 
\ph[width=.80\textwidth]{../cape-of-good-hope/TPO_Images/Western card Down.jpg}{ }


\ph[width=.80\textwidth]{../cape-of-good-hope/TPO_Images/Western- cards posted on the train.jpg}{
28 Dec 1913 Postcard Posted on the train whilst travelling towards Cape Town.
}

\ph[]{../cape-of-good-hope/TPO_Images/Western registered.jpg}{ }


The Travelling Post Ofice service of the Cape of Good Hope commenced late 1882. 
Special vans were fitted to carry mail. 
In 1883 the line reached Victoria West but as the railway lines were 
extended so did the T.P.O.

In 1885 the service reached De Aar and the section from Cape Town to 
De Aaar was named the Western T.P.O.

The vans were coupled to trains which departed from Cape Town at 5.30 p.m. 
every day of the week except Fridays and Saturdays. An express or "fast" 
train left Cape Town on Fridays at 1.15 p.m. Four of the vans proceeded as 
far as Beaufort west and the other two continued to De Aar.

The Western T.P.O. continued to use the no-name postmarks until 
about 1891, the "No Name" T.P.O. was renamed the "WESTERN T.P.O." as it 
operated over the Western route of the Cape Government railways 
between Cape Town and De Aar. It carried on well into the twentieth 
century (until 1950) when T.P.O. were stopped because of expense. 
The Western T.P.O. operated North of De Aar at times and also on 
the route De Aar to Naauwpoort and possibly into the Orange Free State 
on occasion. "UP" in the marks indicated towards Cape Town and 
"DOWN" away from Cape Town. In the course of the train's northbound 
journey from Cape Town, letters, newspapers and parcels were 
dispatched to all offices in the colony, as well as the Orange 
Free State and Transvaal. On the return journey, the main duty 
of the official in charge was to open every mailbag addressed 
to forward post offices 
and to sort the correspondence to facilitate speedy delivery 
on arrival at the terminus.

Mailbags were made up and handed out at stations en route and 
mail for Cape Town was sub-sorted into various categories such 
as government departments, private boxes, poste restante and 
street deliveries. Foreign mail was sub-sorted for transmission 
to London, Manchester, Plymouth, Liverpool, Scotland, Ireland 
and provinces and countries to be served in London. This facilitated 
the dispatch of overseas mail on Wednesdays, which was accomplished 
within three hours of the train's arrival at Cape Town.

 
Western T.P.O. cards posted on the train
28 Dec 1913 Postcard Posted on the train whilst travelling towards Cape Town.
Posting of Letters on the Train

The T.P.O vans carried a supply of postage stamps which the public 
could purchase and mail could be posted on the T.P.O. up to the time 
of the train's departure, or at any intermediate station. This was, 
subject to the payment of 6d. as a late fee over and above the normal 
rate if the mail bags had already been closed. This was denoted by a 
Late Fee handstamp (TPO 25). 

Letters posted on the train are rare 
although a number of them exist and quite possibly more are 
waiting to be found in postcard collections.

 

 
\subsection{Registration Service} 
Western T.P.O. Registered Letter
 
 

A registered service was available 
on the Western T.P.O vans. this registration service was identical 
to those offered in other post offices of the Cape of Good Hope.

 

 

 

 

 

 

 

           