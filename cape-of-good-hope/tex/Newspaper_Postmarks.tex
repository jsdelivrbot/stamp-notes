\section{Cape of Good Hope - Postal History 
} 
\section{Newspaper Branch Postmarks}

\ph[width = .90\textwidth]{../cape-of-good-hope/newspaperbranch/newspaper-postmarks-1.jpg}{Argus Wheels}

 


From approximately 1883, three postmarks (PS 4 to 6) sometimes called Argus Wheels, were in use on circulars as well as newspapers of the Cape of Good Hope.  


\ph[width = .30\textwidth]{../cape-of-good-hope/newspaperbranch/Postmark Cheap Rate.jpg}{
PS 7
}

Another postmark (PS 7) is, in the main, seen stamped on wrappers and circulars, but its use has also been noted on envelopes. The circle measures 27 mm and the upper numerals represent the day and the lower the month. Letters appear on either side of the postmark, but whether these have any significance beyond identifying the handstamp has not been established

The roller cancellers (HRD 1 and 2) were intended to Cheap Rate Postmarkdeface stamps on newspapers at Grahamstown and were used for this purpose until 1864 or '65, when they were replaced by the Barred Oval Numeral Cancellers, generally used for defacement purposes.  The larger post offices were furnished with the cheap rate matter strikes described above. When the Newspaper Branch Office was established at the G.P.O., Cape Town, it was issued with a special datestamp. This, and future handstamps of its type (NP 1 to 5), incorporated the Letters N.P.B. for Newspaper Branch in their design. NP 1, the earliest handstamp in use, is a box with clipped comers. line. A special handstamp (NP 2) was used when the contents came adrift from the newspaper wrapper. It was struck on the wrapper and contained the information "Posted in N.P.B. without contents". The first recorded use of NP 1 and 2 appears to be October 1877. 

<div style="width:85%;margin:0 auto">
<img src="http://localhost/capepostalhistory/newspaperbranch/Envelope%202.jpg" alt="Portrait of De Mist" style="width:90%;display:block;margin:0 auto" />
<p style="font-size:smaller"> 
PS 7 Newspaper mark used on Envelope
</p>
</div> 

Other special datestamps (NP 3 to 5) issued to and used by the Newspaper Branch are similar in design to the normal defacing handstamps of various types but are distinguished by the letters N.P.B. in the lower section of the
circle.

<div style="width:75%;margin:0 auto">
<img src="http://localhost/capepostalhistory/newspaperbranch/Postmarks NP.jpg" alt="Portrait of De Mist" style="width:90%;display:block;margin:0 auto" />
<p style="font-size:smaller"> 
PS 7 Newspaper mark used on Envelope
</p>
</div>

The Letters N.P.B.stand for News Paper Branch. Cape of Good Hope Newspaper Postal Rates used some more information on this postmarks can be found here.

(See also Cheap Rate Postal Markings).

 

  

 

             